\documentclass{beamer}
\usepackage[utf8]{inputenc}
\usepackage{color}
\usepackage{fancyvrb}
\usepackage{listings}
\usetheme{Berlin}
\setbeamerfont{caption}{size=\footnotesize}

\title{Plateforme pour l'analyse technique en Finance}
\subtitle{Soutenance PFE}
\author{CHAUGNY Céline, POINTIN Damien}
\institute{Génie Mathématique | INSA Rouen}

\begin{document}
\lstset{
keywordstyle=\color[rgb]{0,0,1},
commentstyle=\color[rgb]{0.133,0.545,0.133},
stringstyle=\color[rgb]{0.627,0.126,0.941},
}
    \beamertemplatenavigationsymbolsempty

    \begin{frame}
        \titlepage{}
    \end{frame}

    \section*{Sommaire}
        \begin{frame}
            \begin{columns}[t]
  				\begin{column}{5cm}
  					\tableofcontents[sections={1-4},  hideothersubsections]
  				\end{column}
  				\begin{column}{5cm}
  				\tableofcontents[sections={5-9}, hideothersubsections]
  				\end{column}
  			\end{columns}
        \end{frame}

    \section{Introduction}
        \subsection{ }
            
\begin{frame}
    \frametitle{Introduction}    		
    \begin{block}{But du projet}
	Réaliser un programme pour mettre en place une plateforme permettant la gestion d'un portefeuille. Mise en place d'un jeu dans lequel les investisseurs peuvent acheter de vrais actifs de issus de la bourse.
    \end{block}

    \begin{block}{Phases du projet}
	\begin{enumerate}
	 \item Récupérer et stocker certaines données réelles de la bourse.
	 \item Traiter les donner et les analyser avec des outils financiers.
	 \item Permettre à l'utilisateur d'avoir des indicateurs sur ses actifs.
	\end{enumerate}

    \end{block}

\end{frame}

    \section{Technologies utilisées}
        \begin{frame}
            \begin{columns}[t]
  				\begin{column}{5cm}
  					\tableofcontents[sections={1-4}, currentsection, hideothersubsections]
  				\end{column}
  				\begin{column}{5cm}
  				\tableofcontents[sections={5-9}, currentsection, hideothersubsections]
  				\end{column}
  			\end{columns}
        \end{frame}
         \subsection{Dynamic Web Project}
	        \begin{frame}
    \frametitle{Architecture Modèle-Vue-Contrôleur}
    \begin{columns}
  		\begin{column}{5cm}
  		\begin{alertblock}{Vue}
  			Interface Homme-Machine
  		\end{alertblock}
  		\begin{exampleblock}{Contrôleur}
  			Lien entre Modèle et Vue
  		\end{exampleblock}
  		\begin{block}{Modèle}
  			Coeur du programme
  		\end{block}
  		\end{column}
  		
  		\begin{column}{5cm}
  			\includegraphics[scale=0.5]{images/MVC.png}
  		\end{column}
  	\end{columns}
\end{frame}

\begin{frame}
    \frametitle{Environnement Java EE sous Eclipse}
    	  \begin{figure}[H]
      \center
      \includegraphics[scale=0.3]{images/protocoleHTTP.png}
      \caption{Application web - Protocole HTTP}
      \end{figure}

	\begin{block}{IDE Eclipse}
		\begin{itemize}
			\item Gratuit, libre, puissant
			\item Auto-complétion, génération automatique de fonction
			\item Visualisation de la hiérarchie
		\end{itemize}
	\end{block}
\end{frame}

\begin{frame}
    \frametitle{Servlet}
    	  \begin{figure}[H]
      \center
      \includegraphics[scale=0.24]{images/serveurclient.png}
      \end{figure}

	\begin{block}{Configuration des servlets}
		\begin{itemize}
			\item Méthode doGet() et doPost()
			\item Configuration dans web.xml
			\item Apache Tomcat 8.0
		\end{itemize}
	\end{block}
\end{frame}


\begin{frame}[fragile]
    \frametitle{Java Server Pages}
    	  
	\begin{block}{JSP}
		\begin{itemize}
			\item Utilisés pour nos pages web dynamiques
			\item Langage : XML, HTML, Javascript, Java, CSS, ...
		\end{itemize}
	\end{block}
	\begin{block}{Java server page Standard Tag Library}
		\begin{itemize}
			\item Tags prédéfinis
			\item \begin{lstlisting}[language=HTML, basicstyle=\scriptsize] 
<%@ taglib uri="http://java.sun.com/jstl/core" 
prefix="c" %>
<c:out value="Bonjour" />
\end{lstlisting}	  
		\end{itemize}	
			
	\end{block}
\end{frame}

\begin{frame}
    \frametitle{Hiérarchie MVC du projet}
    \begin{columns}
  		\begin{column}{3cm}
  		  	\includegraphics[scale=0.28]{images/exempleMVC.png}

  		\end{column}
  		  		\begin{column}{0.3cm}
  		  		\end{column}

  		\begin{column}{7cm}
  			\includegraphics[scale=0.3]{images/schemaProjetDynamic.png}
  		\end{column}
  	\end{columns}
\end{frame}

	     \subsection{Base de données}
	        \begin{frame}
    \frametitle{MySQL -Driver JDBC}
    \begin{block}{MySQL}
    	\begin{itemize}
    		\item Système de gestion de bases de données relationnelles
    		\item Libre, gratuit, open-source
    	\end{itemize}
    \end{block}
     \begin{block}{Driver Java DataBase Connectivity }
     	\begin{columns}
  				\begin{column}{5cm}
  					\begin{itemize}
  						\item Connexion/Déconnexion
  						\item Requête : Statement 
  						\item Résultat : ResulSet
  					\end{itemize}
  				\end{column}
  				\begin{column}{5cm}
  				      \includegraphics[scale=0.4]{images/jdbc.png}
  				\end{column}
  			\end{columns}

    \end{block}

\end{frame}

\begin{frame}
    \frametitle{Modèle DAO - Principe général}
    \begin{block}{Data Access Object}
    		\begin{itemize}
    			\item Isoler les méthodes concernant le stockage et l'accès aux données
    			\item Création d'une couche supplémentaire :
    		\end{itemize}
    		\center
    		\includegraphics[scale=0.5]{images/dao1.png}
	\end{block}

\end{frame}

\begin{frame}
    \frametitle{Modèle DAO - Mise en place}
     	\begin{columns}
  				\begin{column}{4cm}
  				\begin{block}{Couche DAO}
  				
  					\begin{itemize}
  						\item Méthodes CRUD
  						\item Gestion Exception
  						\item Fabrique 
  						\item Mapping objet-relationnel
  					\end{itemize}
  				\end{block}

  				\end{column}
  				\begin{column}{7cm}
  				      \includegraphics[scale=0.4]{images/dao2.png}
  				\end{column}
  			\end{columns}

\end{frame}    
	    \subsection{Yahoo! Finance}
	        \begin{frame}
    \frametitle{Site Yahoo! Finance}
    
    	   \begin{figure}[H]
      \center
      \includegraphics[scale=0.30]{images/yahoo.png}
      \caption{Visualisation du site Yahoo! Finance pour l'action ACCOR S.A}
      \end{figure}
    	   
\end{frame}

\begin{frame}[fragile]
    \frametitle{Principe du téléchargement}
    \begin{block}{Code JAVA}
  \begin{lstlisting}[language=JAVA, basicstyle=\scriptsize] 
String url="http://real-chart.finance.yahoo.com/table.csv?"+
		           "s="+code  
		           +"&a="+debut.get(Calendar.MONTH) 
		           +"&b="+debut.get(Calendar.DAY_OF_MONTH) 
		           +"&c="+debut.get(Calendar.YEAR) 
		           +"&d="+fin.get(Calendar.MONTH)
		           +"&e="+fin.get(Calendar.DAY_OF_MONTH)
		           +"&f="+fin.get(Calendar.YEAR)
		           +"&g=d" 
		           +"&ignore=.csv"; 
\end{lstlisting}	  
	\end{block}

\end{frame}

	    \subsection{API Google Chart}
	        \begin{frame}
    \frametitle{Présentation de l'API}
    \begin{center}
	      \includegraphics[scale=0.395]{images/google2.png}
	      \hskip1em
	      \includegraphics[scale=0.24]{images/google1.png}
    \end{center}
    \begin{block}{Avantages de l'API}
    	\begin{itemize}
    		\item Graphes intéractifs
    		\item Directement dans la JSP avec JavaScript
    	\end{itemize}
    \end{block}
\end{frame}

\begin{frame} [fragile]
    \frametitle{Présentation de l'API}
    \begin{enumerate}
     \item Charger la librairie :
\begin{lstlisting}[language=HTML, basicstyle=\scriptsize] 
<script type="text/javascript"
src="https://www.gstatic.com/charts/loader.js"></script>
<script type="text/javascript">
google.charts.load('current', {packages: ['corechart']});
google.charts.setOnLoadCallback(drawChart);
</script>
\end{lstlisting}	
     \item Préparer les données : créer une DataTable.
\begin{lstlisting}[language=HTML, basicstyle=\scriptsize] 
var data = new google.visualization.DataTable();
data.addColumn('string', 'Actifs');
data.addColumn('number', 'Quantite');
data.addRows([ ['Obligation', 76.7],['Action', 23.3]]);
\end{lstlisting}
 
    \end{enumerate}
\end{frame}

\begin{frame} [fragile]
    \frametitle{Présentation de l'API}
    \begin{enumerate}
     \setcounter{enumi}{2}
     \item Personnaliser le graphe : titre, dimensions, couleurs,...
\begin{lstlisting}[language=JAVA, basicstyle=\scriptsize]      
var options = { title: 'Repartition portefeuille'};
\end{lstlisting}    
     \item Dessiner le graphe : choix du type de graphe.
\begin{lstlisting}[language=JAVA, basicstyle=\scriptsize]      
var chart = new google.visualization.PieChart(
  document.getElementById('camembert'));
chart.draw(data, options);
\end{lstlisting}    
     \item Afficher le graphe : choisir l'emplacement dans la page HTML.
\begin{lstlisting}[language=HTML, basicstyle=\scriptsize]      
<div id="camembert"></div>
\end{lstlisting}    
    \end{enumerate}
\end{frame}       
        	
     \section{Théorie Finance}
       \begin{frame}
            \begin{columns}[t]
  				\begin{column}{5cm}
  					\tableofcontents[sections={1-4}, currentsection, hideothersubsections]
  				\end{column}
  				\begin{column}{5cm}
  				\tableofcontents[sections={5-9}, currentsection, hideothersubsections]
  				\end{column}
  			\end{columns}
        \end{frame}
        \subsection{Produits financiers}
	        
\begin{frame}
    \frametitle{base}
  

\end{frame}
	    \subsection{Présentation indicateur technique}
	        \begin{frame}
    \frametitle{Présentation générale}
  	\begin{block}{Historique}
  	\begin{itemize}
  		\item Japonais au XVIIIème siècle
  		\item Richard Dow (graphique) puis Nison (chandelier), Bollinger
	\end{itemize}
	
	\end{block}
	\pause
	\begin{block}{Définition}
		\begin{itemize}
			\item John Murphy : " L’analyse technique est l’étude de l’évolution d’un marché, principalement sur la base de graphiques, dans le but de prévoir les futures tendances ".
		\end{itemize}
	\end{block}
\end{frame}

\begin{frame}
    \frametitle{Tendances}
  	\begin{block}{Présentation}
  		Les lignes de tendances sont la loi de base de l’analyse technique :
  		\begin{itemize}
  			\item \textbf{La résistance} rejoint les points hauts de la courbe de valeur du cours.
  			\item \textbf{Le support} de même avec les points les plus bas
  		\end{itemize}
  		\center
  	\includegraphics[scale=0.3]{images/supportresistance.jpg}

	\end{block}
\end{frame}

\begin{frame}
    \frametitle{Chandelier}
    	\begin{block}{Définition}
  		Les chandeliers sont des indicateurs graphiques permettant de représenter la variation d'un cours sur la journée :
  		\begin{itemize}
  			\item \textbf{La chandelier blanc} signifie une hausse du cours.
  			\item \textbf{La chandelier noir} signifie une baisse du cours.
  		\end{itemize}
  		\center
  	\includegraphics[scale=0.3]{images/chandelier.png}
	\end{block}

\end{frame}

\begin{frame}
    \frametitle{Exemples de figures significatives}
       	\begin{columns}
		\begin{column}{3.5cm}
		  \begin{figure}
		      \includegraphics[scale=0.45]{images/chandelier1.png}
		      \caption{Ligne Perçante}		   
		  \end{figure}
		\end{column}
		\begin{column}{3.5cm}
		  \begin{figure}
		      \includegraphics[scale=0.45]{images/chandelier2.png}
		      \caption{Ciel Ouvert}	   
		  \end{figure}
		\end{column}
		\begin{column}{3.5cm}
		  \begin{figure}
		      \includegraphics[scale=0.49]{images/chandelier9.png}
		      \caption{Doji}	   
		  \end{figure}
		\end{column}		
	\end{columns}

\end{frame}

\begin{frame}
    \frametitle{Moyenne mobile}
      	\begin{block}{Définition}
  		\begin{itemize}
  			\item $MM$ : moyenne des cours sur une période donnée.
   			\item $MME = fermeture du jour * 0.09 + MM de la veille * 0.91$. 
		\end{itemize}
	\end{block}
	\begin{figure}
	    \includegraphics[scale=0.3]{images/moyennemobile.png}   
	\end{figure}
\end{frame}


\begin{frame}
    \frametitle{Bollinger}
    \begin{columns}
      \begin{column}{5cm}
	  \begin{block}{Calcul des bandes}
		\begin{itemize}
		\item Moyenne mobile sur une période de n = 20.
		\item Bande supérieure : $MMn + x \times ecartType$
		\item Bande inférieure : $MMn - x \times ecartType$
		\item $ecartType= \sqrt{\sum{\frac{(cloture-MMn)^2}{n}}}$
		\end{itemize}
	  \end{block}
      \end{column}
    \begin{column}{5cm}
	\begin{figure}
	      \includegraphics[scale=0.25]{images/bollinger.png}
	      \caption{Bandes de Bollinger}
	  \end{figure}   
	\end{column}
    \end{columns}
\end{frame}

\begin{frame}
    \frametitle{Bollinger}
    \begin{columns}
      \begin{column}{5cm}
	  \begin{block}{Interprétation}
		\begin{itemize}
		\item Tendance clairement définie, regarder quand le cours intercepte la bande.
		\item Si pas de tendance, regarder quand on croise la bande.
		\item Etranglement : attention !
		\end{itemize}
	  \end{block}
      \end{column}
    \begin{column}{5cm}
	\begin{figure}
	      \includegraphics[scale=0.25]{images/bollingerEtranglement.png}
	      \caption{Exemple d'étranglement.}
	  \end{figure}   
	\end{column}
    \end{columns}
\end{frame}

\begin{frame}
    \frametitle{Volume}
    \begin{columns}
	\begin{column}{4cm}
	  \begin{block}{Définition}
		  \begin{itemize}
			  \item Nombre de titres échangés sur la journée (ou par prix).
			  \item Combiné à d'autres indicateurs.
		  \end{itemize}
	  \end{block}
	\end{column}
	\begin{column}{6cm}
	  \begin{figure}
	      \includegraphics[scale=0.22]{images/volumeBarre.png}   
	  \end{figure}   
	\end{column}
    \end{columns}
\end{frame}
    
		\subsection{Gestion Portefeuille}
	        
\begin{frame}
    \frametitle{Présentation}
    \begin{block}{Définitions}
	\begin{itemize}
	\item \textbf{Gestion de portefeuille :}
	      \begin{itemize}
	      \item Gérer des capitaux sous certaines contraintes
	      \item Choisir une stratégie d'investissement
	      \end{itemize}
	\item \textbf{Différents risques :}
	      \begin{itemize}
	      \item Financiers : marché, crédit, liquidité
	      \item Non financiers : modèle, perte extrême
	      \end{itemize}
	\item \textbf{Profils de rique :}
	      \begin{itemize}
	      \item Risk adverse
	      \item Risk neutral
	      \item Risk lovers/seekers
	      \end{itemize}
	\end{itemize}
    \end{block}


\end{frame}

\begin{frame}
    \frametitle{Actifs risqués}
	  \begin{figure}
	      \includegraphics[scale=0.5]{images/actifsRisques.png}   
	  \end{figure}   
\end{frame}


\begin{frame}
    \frametitle{Diversification}
      \begin{figure}
	  \includegraphics[scale=0.5]{images/exempleChuteEntreprise.png}   
      \end{figure}   
      \begin{block}{Intérêts}
	\begin{itemize}
	 \item Eviter les catastrophes
	 \item Réduire le risque
	\end{itemize}
      \end{block}

\end{frame}
 
	    \subsection{Cadre du projet}
	        
\begin{frame}
    \frametitle{Le broker (courtier)}
      \begin{block}{Définition}
	   \textbf{Broker :} c'est l'intermédiaire d'une ou des opérations financières entre deux parties. Dans le cadre de notre projet, il est l'intermédiaire entre la bourse et les investisseurs (joueurs).
      \end{block}

      \begin{figure}
	  \includegraphics[scale=0.4]{images/schemaProjet.png}
      \end{figure}


\end{frame}
    
	        
	        
	 \section{Modélisation}
        \begin{frame}
            \begin{columns}[t]
  				\begin{column}{5cm}
  					\tableofcontents[sections={1-4}, currentsection, hideothersubsections]
  				\end{column}
  				\begin{column}{5cm}
  				\tableofcontents[sections={5-9}, currentsection, hideothersubsections]
  				\end{column}
  			\end{columns}
        \end{frame}
        \subsection{Cas d'utilisation}
	        \begin{frame}
	\begin{figure}
	    	\includegraphics[scale=0.22]{images/CasDutilisationGeneral.png}

	\end{figure}
\end{frame}
	    \subsection{Diagramme classe}
	        \begin{frame}
    \frametitle{Diagramme Classe Général}
    \begin{figure}
        	\includegraphics[scale=0.25]{images/packages.png}

    \end{figure}
\end{frame}

\begin{frame}
    \frametitle{Diagramme Classe DAO}
    \begin{figure}
    

    	\includegraphics[scale=0.30]{images/packageDAO.png}
    	\end{figure}
\end{frame}

\begin{frame}
    \frametitle{Diagramme Classe Controleur}
    \begin{figure}

    	\includegraphics[scale=0.30]{images/packageControleur.png}
    	    	\end{figure}

\end{frame}

\begin{frame}
    \frametitle{Diagramme Classe Modele}
        \begin{figure}

    	\includegraphics[scale=0.20]{images/DiagrammeClasseFinalModele.png}
    	    	\end{figure}

\end{frame}

\begin{frame}
    \frametitle{Diagramme Architecture WebContent}

    \begin{figure}
    	\includegraphics[scale=0.25]{images/packagesWebContent.png}
    	    	\end{figure}

\end{frame}    
		\subsection{Diagramme séquence}
	        \begin{frame}
    \frametitle{Diagramme Séquence Inscription}
    \begin{figure}
		\includegraphics[scale=0.3]{images/DiagrammeSequenceInscription.png}
	\end{figure}

\end{frame}

\begin{frame}
    \frametitle{Diagramme Séquence Connexion}
    \begin{figure}
		\includegraphics[scale=0.3]{images/DiagrammeSequenceConnexion.png}
	\end{figure}

\end{frame}

\begin{frame}
    \frametitle{Diagramme Séquence Achat Actif}
    \begin{figure}
		\includegraphics[scale=0.23]{images/DiagrammeSequenceAchatActif.png}
	\end{figure}

\end{frame}

\begin{frame}
    \frametitle{Diagramme Séquence Achat Actif (Zoom modele)}
    \begin{figure}
		\includegraphics[scale=0.28]{images/DiagrammeSequenceAchatActifModele.png}
	\end{figure}

\end{frame}

\begin{frame}
    \frametitle{Diagramme Séquence Vente Actif}
    \begin{figure}
		\includegraphics[scale=0.23]{images/DiagrammeSequenceVenteActif.png}
	\end{figure}

\end{frame}

\begin{frame}
    \frametitle{Diagramme Séquence Supprimer Portefeuille}
    \begin{figure}
		\includegraphics[scale=0.22]{images/DiagrammeSequenceSupprimerPortefeuille.png}
	\end{figure}

\end{frame}
	    \subsection{Base de données}
	        
\begin{frame}
    \frametitle{Yahoo}
  

\end{frame}
	    \subsection{Modélisation de la bourse et portefeuille}
	        \input{tex/bourse.tex}
    
    \section{Travail de groupe}
        \begin{frame}
            \begin{columns}[t]
  				\begin{column}{5cm}
  					\tableofcontents[sections={1-4}, currentsection, hideothersubsections]
  				\end{column}
  				\begin{column}{5cm}
  				\tableofcontents[sections={5-9}, currentsection, hideothersubsections]
  				\end{column}
  			\end{columns}
        \end{frame}
        \subsection{Répartition}
	        
\begin{frame}
    \frametitle{Division du travail}
     \begin{block}{Ensemble}
      \begin{itemize}
       \item Modélisation générale du projet
       \item Choix des outils techniques
       \item Modélisation du modèle, de la base données
      \end{itemize}

     \end{block}
     \begin{block}{Séparemment}
      \begin{itemize}
       \item Séparation des technologies : mise en place de la base de données, des pages web, ...
       \item Séparation de portions de code
      \end{itemize}
     \end{block}
\end{frame}
	    \subsection{Git}
	        cd 
\begin{frame}
    \frametitle{Plateforme git}
    		\begin{block}{Git}
    			\begin{itemize}
    				\item Logiciel de gestion de versions décentralisé
    				\item Hébergé sur monprojet.insa-rouen.fr puis github
    				\item Au début, beaucoup de difficultés à le configurer (fichier gitignore) 
    			\end{itemize}
    		\end{block}
  

\end{frame}    
	
	  \section{Démonstration}
        \begin{frame}
            \begin{columns}[t]
  				\begin{column}{5cm}
  					\tableofcontents[sections={1-4}, currentsection, hideothersubsections]
  				\end{column}
  				\begin{column}{5cm}
  				\tableofcontents[sections={5-9}, currentsection, hideothersubsections]
  				\end{column}
  			\end{columns}
        \end{frame}
        \subsection{Démo}
	        
\begin{frame}
   Démonstration
  

\end{frame}
        	
    \section{Conclusion}
        \subsection{}
            \begin{frame}
    \frametitle{Conclusion}
    \begin{block}{Conclusion}
    		Ce projet nous a permis 
    	\begin{itemize}
    		\item d'apprendre à gérer un travail de groupe
    		\item de découvrir de nouvelles technologies et de nouveaux outils
    		\item d’approfondir nos connaissances en finance
    	\end{itemize}
    \end{block}
\end{frame}

\end{document}
