\section{Gestion de portefeuille}
https://fr.wikipedia.org/wiki/Gestion\_d'actifs

\subsection{Définitions et généralités}

Un \textbf{portefeuille} est un ensemble homogène de ressources ou d'actifs. En finance, un portefeuille est composé d'actfis financiers qui peuvent être de différentes natures (actions, obligations, options,...).\\

La \textbf{gestion de portefeuille} consiste à :
\begin{itemize}
 \item Gérer des capitaux tout en respectant les contraintes réglementaires et contractuelles qu'on peut avoir.
 \item Appliquer les politiques d'investissements définies au départ par le propriétaire du portefeuille.
 \item Tirer le meileur rendement possible en fonction du niveau de risque choisi par l'investisseur.
\end{itemize}

Afin d'illuster la gestion de portefeuille en général et sans rentrer tout de suite dans le cadre de la gestion d'un portefeuille d'actifs financiers, voyons les exemples suivants qui illustrent différents objectifs :
\begin{enumerate}
 \item Un individu seul voudra par exemple préparer sa retraite ou bien simplement investir en bourse plutôt que laisser son argent dormir sur un compte qui rapport peu.
 \item Les banques elles auront pour but de faire fructifier les dépôts de ses clients pour pouvoir verser les intérêts et surtout avoir suffisamment de liquidités en cas de retraits massifs de ces même dépôts.
 \item Les assureurs quant à eux auront pour objectifs d'assurer le paiement des sinistres et, en cas de situations exceptionnelles, ils auront besoin de liquidités (exemple : catastrophe naturelle).
\end{enumerate}


\subsubsection{Les différents risques}
On identifie deux grands types de risques lorsque l'on investi dans un portefeuille d'actifs financiers :
\begin{enumerate}
 \item \textbf{Les risques financiers :} lorsque l'on investi sur les marchés financiers, on est par définition exposé aux risques qu'ils comportent.
    \begin{itemize}
     \item Le risque de marché : il existe une incertitude en ce qui concerne les taux de marché, les prix des actifs (actions, devises,...).
     \item Le risque de crédit, de contrepartie, de défaut : on n'est pas à l'abri du fait que la personne en face remplisse ses obligations, sauf si l'on est dans le cadre d'un marché organisé.
     \item Le risque de liquidité : il est possible qu'on soit obligé de vendre un actif à un prix plus faible que sa juste valeur, on perd alors de l'argent.
    \end{itemize}
 \item \textbf{Les risques non-financiers :} il en existe énormément, nous ne citerons que ceux qui peuvent vraiment avoir un impact dans la situtation d'un individu qui investi seul (et non pas la gestion d'un portefeuille pour une entreprise telle qu'une banque ou un assureur).
    \begin{itemize}
     \item Le risque de modèle : on peut se tromper de modèle d'évaluation des actifs, d'analyse ou d'aide à la décision.
     \item Le risque de liquidité : il est possible qu'on soit obligé de vendre un actif à un prix plus faible que sa juste valeur, on perd alors de l'argent.
     \item Le risque de perte extrême : c'est par exemple le risque qu'une entreprise pourtant très solide s'effondre.
    \end{itemize}
\end{enumerate}
Il est important de préciser que les risques ne sont pas indépendants, c'est pourquoi il faut être d'autant plus vigilant.

\subsubsection{Les mesures du risque}
\begin{itemize}
 \item Le rendement :
 \item La volatilité :
\end{itemize}



\subsubsection{La diversification d'un portefeuille}
Diversifier son portefeuille est la première chose à faire lorsque l'on souhaite éviter le risque de perte du capital investi. On peut justifier ceci à l'aide de l'exemple suivant :

Tout d'abord, cela permet d'éviter les catastrophes.
\paragraph{Exemple :} \textit{Un individu qui investi dans une seule entreprise depuis 10 ans pour préparer sa retraite, le cours de l'action ne fait qu'augmenter, mais il peut arriver que l'entreprise subisse un 'coup dur' et alors son cours va chuter de manière rapide et critique. L'individu aura alors tout perdu.
	  S'il avait investi dans un portefeuille, il aurait été peu probable que tous ses actifs chutent en même temps (sauf en cas de crise économique répandue) et la chute d'une entreprise aurait eu des conséquences moindres sur son investissement.}\\
La conclusion de cet exemple est qu'il vaut mieux de pas avoir un seul titre dans son portefeuille.

Cela permet également de réduire le risque sauf dans certains cas (crise économique).


\subsubsection{Processus de gestion d'un portefeuille}
\begin{enumerate}
 \item Planifier : comprendre les besoins de l'investisseur, analyser sa tolérance au risque et préparer une politique de placement (objectifs, contraintes).
 \item Exécuter : allouer des actifs au portefeuille, les analyser et construire le portefeuille.
 \item Faire un bilan : mesurer les performances du portefeuille et le modifier ou l'ajuster en fonction des résultats.
\end{enumerate}


\subsubsection{Les différents profils de risque}
\begin{itemize}
 \item \textbf{Les risk lovers/seekers :}
 \item \textbf{Les risk neutral :}
 \item \textbf{Les risk adverse :}
\end{itemize}


Exemple bidon :
Même rendement et volatilités différentes : on choisit la volatilité la plus faible si on est censé.
Même volatilité et rendements différents : meilleur rendement.


	  
	  
	  Actifs risqués et non risqués
Analyse d'un historique : calculs rendement actif
Analyse d'un historique : variance (volatilité)
Covariance entre deux actifs
Rendement d'un portefeuille
Optimisation d'un portefeuille : Markowitz


la VaR?

\subsection{La théorie moderne du portefeuille}
