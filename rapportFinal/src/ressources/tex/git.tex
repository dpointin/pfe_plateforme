\section{Utilisation de git}
Git est un logiciel qui permet la gestion des versions d'un projet. Nous avons utilisé ce logiciel pour nous permettre de pouvoir travailler sur notre projet de manière synchronisée. Dans un premier temps, nous avons hébergé notre projet sur monprojet.insa-rouen.fr puis sur github suite à quelques problèmes de serveur sur monprojet.insa-rouen.fr. \\

Nous avons utilisé principalement quatre commandes git : 
\begin{itemize}
\item git add qui permet d'ajouter des modifications à notre branche. 
\item git commit qui permet d'indexer nos modifications
\item git push qui permet de mettre notre branche comme étant la branche courante sur le serveur
\item git pull qui nous permet de récupérer les modifications sur la brancher master
\end{itemize}

Nous avons quelques difficultés au début lors de l'utilisation de Git : en effet nous ne connaissions pas les fichiers .gitignore qui permettent d'ignorer des fichiers dans le dépot git. Nous avions donc mis tout notre workspace dans le dépot et cela posait des problèmes car nous avions mis nos fichiers de configuration. Ainsi il y avait tous les fichiers qui permettent de configurer le build path ainsi que le serveur. Le problème est que nous n'avions pas la même arborescence donc il nous fallait reconfigurer le workspace à chaque utilisation. 


\section{Répartition du travail}

Nous avons dans un premier temps effectué des recherches sur les différents outils d'analyse technique, cette première partie de documentation a été effectuée chacun de son côté. \\

Ensuite la deuxième phase de notre projet a été la définition des besoins et la réalisation de la modélisation. Nous avons réalisé cette partie ensemble car il fallait que nous soyons sur d'avancer dans la même direction avant de passer à l'implémentation. Il nous a fallu plusieurs semaines pour réussir à réaliser une première modélisation qui nous semblait pertinente pour passer à la suite. \\

Ensuite, il nous a fallu étudier les outils que nous pouvions utiliser pour réaliser notre projet. Nous avions décidé plutôt rapidement de faire une application web grâce à Java EE car nous l'avions déjà utilisé en cours l'année précédente. Il nous a ensuite fallu étudier les différentes technologies qui nous permettait de tracer les graphes, crypter le mot de passe ou connecter notre base de données. Nous avons regardé les différentes technologies ensemble pour choisir la plus pertinente. Une fois le choix réalisé, chacun s'est "spécialisé" dans plusieurs technologies. Ainsi Céline, a regardé davantage les technologies JSTL, CSS, JavaScript et l'API Google Chart alors Damien a regardé d'avantage Jasypt et le modèle DAO. \\

Ensuite, nous avons du avancer dans l'implémentation à proprement parler. Nous avons commencé l'implémentation ensemble avant d'utiliser l'outil Git. Comme dit précédemment nous avons eu quelques soucis lors de son utilisation au début. Cela nous a donc fait perdre beaucoup de temps lors de la réalisation des premières versions. Une fois le problème réglé, nous avons pu vraiment avancer. Lors de réunion hebdomadaires nous décidions les différents points à avancer en priorité pour la semaine suivante ainsi que les différents problèmes que chacun a rencontré pour essayer de les résoudre ensemble si le problème persistait. 
