\section{Division des tâches}

\subsection{Les méthodes agiles}
Les méthodes agiles sont des groupes de pratiques de projets en développement en informatiques.\\
\\
Elles reposent sur quatre valeurs importantes : \\
\begin{itemize}
\item{L'équipe : il faut une bonne communication entre les membres de l'équipe}
\item{L'application : il faut qu'elle soit fonctionnelle et que la documentation technique soit mise à jour}
\item{La collaboration : le client est impliqué dans le déroulement}
\item{L'acceptation du changement : la planification réalisée est flexible}
\end{itemize} 
~\\
~\\
Les différentes étapes à suivre sont : \\
\begin{itemize}
\item{Le responsable fonctionnel définit et ordonne la production des composants de l'application}
\item{Le projet est structuré en incréments de 1 à 6 semaines suivant les nécessités}
\item{Une réunion initiale organise chaque incrément en définissant les tâches à réaliser}
\item{Chaque jour, courte réunion pour donner à l'équipe une vision globale du projet : avancement, problème et solution}
\item{Reporting mural mis à jour et en temps réel par les membres de l'équipe}
\item{Un incrément est terminé s'il est complet, développé, approuvé, testé et documenté}
\item{Réunion finale pour chaque incrément}
\item{Validation du travail de l'équipe par le responsable fonctionnel}
\end{itemize}

Nous n'avons pas mis en place les méthodes agiles dès le debut du projet ce qui a été préjudiciable dans son avancé. A la fin, nous avions un déroulement proche des méthodes agiles. Chaque semaine nous nous retrouvions pour faire un bilan de notre avancée, faire un bilan des objectifs atteints et restant à atteindre et définir les objectifs pour la semaine suivante. 
