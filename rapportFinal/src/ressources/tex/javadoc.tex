\section{Documentation Technique}

La documentation technique se décompose en deux partie principales : 

\begin{itemize}
\item La documentation technique pour un utilisateur du programme, il s'agit en fait d'un manuel d'utilisation détaillé. Il faut que cette documentation présente toutes les fonctionnalités de l'application
\item La documentation technique pour un futur développeur du programme, il s'agit d'une vision plus détaillée du code. Cette documentation permet à un futur développeur de prendre en main plus facilement le code en sachant ce que chaque fonction prend en entrée, modifie et renvoie en sortie par exemple. 
\end{itemize}

Le chapitre 5 de ce rapport, Utilisation de l'application correspond à une documentation technique pour l'utilisateur. \\

Pour la documentation technique à destination d'un futur programmeur nous avons choisi d'utiliser Javadoc. Javadoc est un outil développé par Sun Microsystems qui permet de générer une documentation technique depuis les commentaires (présents entre certaines balises) du code source en Java au format HTML. \\

Javadoc présente l'avantage d'être un standard industriel pour la documentation technique en Java. L'avantage de Javadoc est que la documentation se trouve extraite du code source lui même ce qui permet de consulter son code lors de l'écriture. Cela permet de réaliser facilement la documentation au fur et à mesure de la rédaction du code. Cependant certains voient comme inconvénient à Javadoc le fait qu'elle ne puisse être rédigé que par les programmeurs. 