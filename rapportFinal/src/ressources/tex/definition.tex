\section{Définitions}

\subsection{Les marchés}

Lorsqu'un acheteur et un vendeur souhaitent conclure une transaction, ils peuvent le faire sur deux types de marché : le marché organisé ou le marché gré à gré.  

\subsubsection{Marché organisé}
Dans un marché organisé, l'acheteur et le vendeur placent des ordres d'achats et de vente via une société de bourse telle que NYSE Euronext par exemple. \\

Une société de bourse est une société commerciale ayant pour but d'organiser le marché financier en fixant des règles de fonctionnement et d'admission au marché. Ces règles doivent respecter celles définies par l'autorité des marchés financiers (AMF) en France. Le marché organisé n'est donc pas ouvert à tous, les membres de ce marché peuvent ainsi transmettre les ordres de leurs clients, qu'ils soient particuliers ou institutionnels. Un membre de ce marché peut être par exemple un courtier. \\

Sur un marché organisé il existe une chambre de compensation qui a pour objectif de garantir la finalité des opérations aux vendeurs et acheteurs. Pour pouvoir participer au marché organisé il faut payer des frais de gestion. 

\subsubsection{Marché gré à gré}
Dans un marché gré à gré, la transaction est conclue bilatéralement entre les deux parties. \\

Dans ce marché, les règles sont donc plus souples car il est possible de discuter directement avec l'acheteur (respectivement le vendeur) pour définir toutes les clauses du contrat comme un paiement échelonné par exemple.  

\subsubsection{Marché primaire}
C'est un lieu où sont émis les nouveaux titres.

\subsubsection{Marché primaire}
C'est le lieu ou l'acheteur peut revendre ses titres à un cours défini par la rencontre de l'offre et la demande. 

\subsection{Les différents produits financiers}

\subsubsection{Action}
Une action représente une part du \textbf{capital} de la société émettrice. Dés qu'une société est cotée en bourse, les actions qui composent son capital social vont évoluer en fonction des achats et des ventes des investisseurs en suivant le jeu de l'offre et de la demande.  \\
En achetant une action, on fait grimper le prix. Au bout d’un moment, le prix du marché de l’action va dépasser sa fair value, on décide alors de la vendre pour cristalliser le bénéfice gagné. En vendant une action, on fait alors baisser le prix.\\ 

Le détenteur d'actions est qualifié d'actionnaire et l'ensemble des actionnaires constitue l'actionnariat. \\


Les action donnent des droits à leurs propriétaires : 
\begin{itemize}
\item Droits politiques : droit à l'information et au vote
\item Droits financiers : droit aux dividendes
\end{itemize}

\subsubsection{Indice boursier}

\subsubsection{Obligation}
Une obligation est représentative d'une partie de la dette d'un émetteur à moyen ou long terme. Cette dette est émise dans une devise donnée, pour une durée définie et donne le droit au paiement d'un intérêt fixe ou variable appelé \textbf{coupon} qui peut être capitalisé jusqu'à maturité. L'émetteur d'un action est \textbf{l'emprunteur} alors que le porteur d'une obligation est le \textbf{créancier}. \\

En 2014 le marché obligataire mondiale représentait 150 trillion de dollars soit plus de 50\% du marché total des actifs financiers. \\

Il peut y avoir divers émetteurs :
\begin{itemize}
\item Un état dans sa propre devise, on parle \textbf{d'emprunt d'état}
\item Un état dans une autre autre devise, on parle \textbf{d'obligation souverain}
\item Une entreprise du secteur public, on parle \textbf{d'obligation du secteur public}
\item Une entreprise du secteur privée, association on parle \textbf{d'obligation corporative}
\end{itemize}

Pour mesurer le risque lié à l'émetteur de l'obligation des agences de notation attribuent une note aux émetteurs qui en font la demande. Par exemple pour Moody's, un émetteur noté Aaa est de qualité supérieur (équivalent de AAA chez Standard and Poors et Fitch) alors qu'un émetteur noté C est très proche de la faillite. La notation va avoir un impact sur le rendement de obligation, en effet si on choisit de prêter à quelqu'un qui n'est pas fiable on attend en contrepartie un taux d'intérêt élevé. Au contraire, si on choisit de prêter à une entreprise ou un état de qualité supérieur on attend que très peu de bénéfice. 

\textbf{Cas particulier : } Une obligation convertible est un type particulier d'obligation. En effet, à une date déterminée le détenteur de l'obligation a le droit (et non l'obligation) de convertir son obligation en action de l'entreprise. C'est-à-dire de transformer une partie de la dette de l'entreprise qu'il détenait en part dans l'entreprise. 

\subsubsection{Matière première}

\subsubsection{Option}
Une option est un produit dérivé (c'est à dire dire un produit financier dont prix dépend d'un actif, appelé sous-jacent) qui établit un contrat entre un vendeur et un acheteur. \\

Une option donne le droit à l'acheteur (le vendeur quand à lui n'a pas le choix) :
\begin{itemize}
\item d'acheter (option call)
\item de vendre (option put) 
\end{itemize}

une quantité donnée de l'actif sous-jacent à un prix fixé à l'avance aussi appelé \textbf{stike}. Cette transaction à lieu à une date donnée, maturité, si l'option est \textbf{européenne} ou durant toute la période jusqu'à la maturité si l'option est \textbf{américaine}. Ce droit d'achat ou de vente ce négocie contre un certain prix \textbf{la prime}. \\

On distingue la position longue (acheteur) de la position courte (vendeur) d'une option. Une option est levée si l'acheteur de l'option est gagnant, si ce n'est pas le cas l'option est dite abandonnée.

Grâce aux options on peut effectuer les stratégies suivantes : 
\begin{itemize}
\item acheter un call pour jouer sur la hausse du cours de l'actif sous-jacent
\item acheter un put pour jouer sur la baisse du cours de l'actif sous-jacent
\item vendre un call pour jouer sur la baisse du cours de l'actif sous-jacent
\item vendre un put pour jouer sur la hausse du cours de l'actif sous-jacent
\end{itemize}

On distingue trois grand type d'option :
\begin{itemize}
\item Les options vanilles, les plus simples étant des calls, des puts ou des combinaisons de call et put tel que le stradlle.
\item Les options de première génération, utilisée principalement généralement sur le marché des taux d'intérêt. Par exemple le cap plafonne le taux d'emprunt et le floor limite le taux. 
\item Les options de seconde génération sont plus flexibles. Par exemple les options binaires garantissent un gain fixe à l'acheteur si l'actif sous-jacent est à niveau supérieur au prix d'exercice lors d'un call. 
\end{itemize}

\subsection{Les différents acteurs}

\subsubsection{Investisseurs institutionnels}

\subsubsection{OPCVM}

\subsubsection{Broker}

\subsubsection{Hedge Funds}

\subsubsection{Banque d'investissement}