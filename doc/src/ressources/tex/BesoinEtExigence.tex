\chapter{Besoin et exigences}
Notre projet va donc se découper en trois phases principales : 
\begin{itemize}
\item La première consistera à récupérer les données des cours et les stocker par l’intermédiaire d’une base de données.
\item La deuxième étape sera le traitement de ces données (outils d’analyse technique).
\item Enfin, nous afficherons à l’utilisateur un tableau de bord. 
\end{itemize}

\section{Base de données}
Gestion des différents cours des actifs (Nom, valeur à l’ouverture, à la clôture, la plus haute et la plus basse pendant la séance). \\ \\ 
Deux options pour la récupération des données : soit on les stocke dans une base de donnée statique, c’est-à-dire, nous aurons récupéré les cours pour une certaine durée. Nous placerons le début du jeu dans le passé, de manière à pouvoir constater les gains ou pertes du joueur. La deuxième méthode pour gérer nos données, consisterait à actualiser notre base de données à la demande du joueur en allant chercher les données sur un serveur web. 

\section{Gestion des actifs}
Pour chaque actif nous proposerons une analyse des divers cours que nous aurons dans notre base de données. Pour cela nous utiliserons des outils d’analyse technique. \\ \\
A la fin de l’analyse, nous proposerons un avis sur l’actif. 

\section{Tableau de bord}
Chaque utilisateur aura la possibilité de consulter l’ensemble de son portefeuille. Pour cela un descriptif lui sera affiché avec les fonctionnalités suivantes : détails (composition du portefeuille, liste des actifs), performance et la répartition par titres. \\ \\
Ensuite pour chaque action, il sera possible de visualiser la valeur du cours, sa variation et son volume, un graphique et plusieurs possibilités d’indicateurs. 