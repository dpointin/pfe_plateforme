\section{Définitions}

\subsection{Les marchés}

Lorsqu'un acheteur et un vendeur souhaitent conclure une transaction, ils peuvent le faire sur deux types de marché : le marché organisé ou le marché gré à gré.  

\subsubsection{Marché organisé}
Dans un marché organisé, l'acheteur et le vendeur placent des ordres d'achats et de vente via une société de bourse telle que NYSE Euronext par exemple. \\

Une société de bourse est une société commerciale ayant pour but d'organiser le marché financier en fixant des règles de fonctionnement et d'admission au marché. Ces règles doivent respecter celles définies par l'autorité des marchés financiers (AMF) en France. Le marché organisé n'est donc pas ouvert à tous, les membres de ce marché peuvent ainsi transmettre les ordres de leurs clients, qu'ils soient particuliers ou institutionnels. Un membre de ce marché peut être par exemple un courtier. \\

Sur un marché organisé il existe une chambre de compensation qui a pour objectif de garantir la finalité des opérations aux vendeurs et acheteurs. Pour pouvoir participer au marché organisé il faut payer des frais de gestion. 

\subsubsection{Marché gré à gré}
Dans un marché gré à gré, la transaction est conclue bilatéralement entre les deux parties. \\

Dans ce marché, les règles sont donc plus souples car il est possible de discuter directement avec l'acheteur (respectivement le vendeur) pour définir toutes les clauses du contrat comme un paiement échelonné par exemple.  

\subsubsection{Marché primaire}
C'est un lieu où sont émis les nouveaux titres.

\subsubsection{Marché primaire}
C'est le lieu ou l'acheteur peut revendre ses titres à un cours défini par la rencontre de l'offre et la demande. 

\subsection{Les différents produits financiers}

\subsubsection{Action}
Une action représente une part du \textbf{capital} de la société émettrice. Dés qu'une société est cotée en bourse, les actions qui composent son capital social vont évoluer en fonction des achats et des ventes des investisseurs en suivant le jeu de l'offre et de la demande.  \\
En achetant une action, on fait grimper le prix. Au bout d’un moment, le prix du marché de l’action va dépasser sa fair value, on décide alors de la vendre pour cristalliser le bénéfice gagné. En vendant une action, on fait alors baisser le prix.\\ 

Le détenteur d'actions est qualifié d'actionnaire et l'ensemble des actionnaires constitue l'actionnariat. \\


Les action donnent des droits à leurs propriétaires : 
\begin{itemize}
\item Droits politiques : droit à l'information et au vote
\item Droits financiers : droit aux dividendes
\end{itemize}

\subsubsection{Indice boursier}
Un indice boursier est un panier d'actions dont les variations sont supposées refléter au mieux les fluctuations d'un marché (par exemple le marché français) ou d'un secteur d'activité particulier (par exemple l'indice sectoriel du secteur aéronautique). \\

Un indice boursier est un outil statistique qui est calculé à partir de la moyenne des titres qui le composent. En général, cette moyenne est pondéré : certains actifs ont un poids plus fort. Cette pondération peut-être fonction de la capitalisation boursière par exemple (nombre d'actions émises multipliées par leurs valeurs. Euronext (principal opérateur financier de la zone euro) permet de prendre connaissance et de suivre tous les indices boursiers de la zone euro.  \\

Les principaux indices boursiers du marché français sont les suivants :
\begin{itemize}
\item Le CAC40 (Cotation Assistée en Continu est l'indice de référence de la place de Paris. Il a été créé en 1998 avec une base de 1000 points et est constitué des quarante plus grandes valeurs cotées en continu à la bourse de Paris. 
\item Le SBF 120 (Société des Bourses Françaises) est composé des 40 valeurs du CAC40, des 20 valeurs de l'antichambre du CAC40 (CAC Next 20) et des 60 valeurs les plus liquides de la bourse de Paris.
\item CAC All-Tradable a remplacé le SBF 250 en 2011 et représente les 250 valeurs les plus importantes de la bourse de Paris.  
\end{itemize}

Les principaux indices boursiers mondiaux : 
\begin{itemize}
\item Le Dow Jones est le plus vieil indice boursier du monde, il est l'indice le plus important de New-York. Il est composé des 30 valeurs américaines les plus importantes. 
\item Le Nasdaq est un indice de Wall Street également composé essentiellement de valeurs technologiques. 
\item Le Dax30 est le principal indice de la banque de Francfort, il est composé des 30 valeurs cotées à Francfort les plus importantes.
\item Le Footsie 100 est le principal indice de la bourse de Londres qui repose sur les 100 plus grandes valeurs cotées à la City. 
\end{itemize}

\subsubsection{Obligation}
Une obligation est représentative d'une partie de la dette d'un émetteur à moyen ou long terme. Cette dette est émise dans une devise donnée, pour une durée définie et donne le droit au paiement d'un intérêt fixe ou variable appelé \textbf{coupon} qui peut être capitalisé jusqu'à maturité. L'émetteur d'une obligation est \textbf{l'emprunteur} alors que le porteur d'une obligation est le \textbf{créancier}. \\

En 2014 le marché obligataire mondiale représentait 150 trillion de dollars soit plus de 50\% du marché total des actifs financiers. \\

Il peut y avoir divers émetteurs :
\begin{itemize}
\item Un état dans sa propre devise, on parle \textbf{d'emprunt d'état}
\item Un état dans une autre autre devise, on parle \textbf{d'obligation souveraine}
\item Une entreprise du secteur public, on parle \textbf{d'obligation du secteur public}
\item Une entreprise du secteur privée, association on parle \textbf{d'obligation corporative}
\end{itemize}

Pour mesurer le risque lié à l'émetteur de l'obligation des agences de notation attribuent une note aux émetteurs qui en font la demande. Par exemple pour Moody's, un émetteur noté Aaa est de qualité supérieur (équivalent de AAA chez Standard and Poors et Fitch) alors qu'un émetteur noté C est très proche de la faillite. La notation va avoir un impact sur le rendement de obligation, en effet si on choisit de prêter à quelqu'un qui n'est pas fiable on attend en contrepartie un taux d'intérêt élevé. Au contraire, si on choisit de prêter à une entreprise ou un état de qualité supérieur on attend que très peu de bénéfice. 

\textbf{Cas particulier : } Une obligation convertible est un type particulier d'obligation. En effet, à une date déterminée le détenteur de l'obligation a le droit (et non l'obligation) de convertir son obligation en action de l'entreprise. C'est-à-dire de transformer une partie de la dette de l'entreprise qu'il détenait en part dans l'entreprise. 

\subsubsection{Matière première}
Une matière première est un matériau, une denrée ou une substance intervenant dans la production des biens intermédiaires et des produits finis. Il s'agit de matières produites par la nature qui nécessitent une transformation pour utilisation. Les matières premières peuvent être achetées et vendues sur les Bourses de commerce du monde entier.\\

Nous pouvons regrouper les matières premières en plusieurs grandes catégories :\\
\begin{itemize}
\item \textbf{Les matières premières énergétiques}. Dans cette catégorie, nous retrouvons par exemple le pétrole (négocié sous forme de pétrole brut ou de produits raffinés) ou le gaz naturel. 
\item \textbf{Les matières premières agricoles} Dans cette catégorie nous retrouvons toutes les denrées alimentaires tels que le maïs ou les blé. Leur prix est dépendant des conditions climatiques.
\item \textbf{Les métaux précieux} On peut penser à l'or ou l'argent par exemple. 
\end{itemize}


\subsubsection{Option}
Une option est un produit dérivé (c'est à dire dire un produit financier dont prix dépend d'un actif, appelé sous-jacent) qui établit un contrat entre un vendeur et un acheteur. \\

Une option donne le droit à l'acheteur (le vendeur quand à lui n'a pas le choix) :
\begin{itemize}
\item d'acheter (option call)
\item de vendre (option put) 
\end{itemize}

une quantité donnée de l'actif sous-jacent à un prix fixé à l'avance aussi appelé \textbf{stike}. Cette transaction à lieu à une date donnée, maturité, si l'option est \textbf{européenne} ou durant toute la période jusqu'à la maturité si l'option est \textbf{américaine}. Ce droit d'achat ou de vente ce négocie contre un certain prix \textbf{la prime}. \\

On distingue la position longue (acheteur) de la position courte (vendeur) d'une option. Une option est levée si l'acheteur de l'option est gagnant, si ce n'est pas le cas l'option est dite abandonnée.

Grâce aux options on peut effectuer les stratégies suivantes : 
\begin{itemize}
\item acheter un call pour jouer sur la hausse du cours de l'actif sous-jacent
\item acheter un put pour jouer sur la baisse du cours de l'actif sous-jacent
\item vendre un call pour jouer sur la baisse du cours de l'actif sous-jacent
\item vendre un put pour jouer sur la hausse du cours de l'actif sous-jacent
\end{itemize}

On distingue trois grand type d'option :
\begin{itemize}
\item Les options vanilles, les plus simples étant des calls, des puts ou des combinaisons de call et put tel que le stradlle.
\item Les options de première génération, utilisée principalement généralement sur le marché des taux d'intérêt. Par exemple le cap plafonne le taux d'emprunt et le floor limite le taux. 
\item Les options de seconde génération sont plus flexibles. Par exemple les options binaires garantissent un gain fixe à l'acheteur si l'actif sous-jacent est à niveau supérieur au prix d'exercice lors d'un call. 
\end{itemize}

\subsection{Les différents acteurs}

\subsubsection{Investisseurs institutionnels}
Les investisseurs institutionnels, également appelés grands investisseurs, sont des organismes collecteurs de l'épargne qui placent leurs fonds sur les marchés. Il s'agit principalement de sociétés d'investissement, de fonds de pension, d'organismes de placement collectif en valeurs mobilières (OPCVM) ou de sociétés d'assurance. \\

Il existe principalement deux types d'OPCVM : les Sociétés d'investissement à capital variables (Sicav) ou les Fonds communs de placement (Fcp). 

\subsubsection{Hedge Funds}
Le terme Hedge Fund englobe une grande variété de fonds utilisant des techniques de gestion non traditionnelle. Le premier hedge fund a été constitué en 1949 aux États-Unis par un sociologue et journaliste financier : Alfred Wislow Jones. \\

 Les hedge funds sont des fonds d'investissement non cotés à vocation spéculative. Ce sont des fonds spéculatifs recherchant des rentabilités élevées et qui utilisent abondamment les produits dérivés, en particuliers les options. \\

Les hedge funds présentent l'intérêt d'offrir une diversification supplémentaire aux portefeuilles « classiques » car leurs résultats sont en théorie déconnectés des performances des marchés d'actions et d'obligations. \\


\subsubsection{Banque d'investissement}
Une banque d'investissement est une banque, ou une division de la banque, qui rassemble l'ensemble des activités de conseil, d'intermédiation et d'exécution ayant trait aux opérations dites de haut de bilan (introduction en Bourse, émission de dette, fusion/acquisition) de grands clients (entreprises, investisseurs, mais aussi États…). \\

Ces activités sont généralement scindées en entités distinctes, habituellement désignées par des anglicismes : les opérations de Corporate Finance (finance d'entreprise), de Global Capital Markets (marchés financiers), et de Structured Finance (opérations de financement). \\

On différencie parfois la banque d'investissement de la banque d'affaires en attribuant à la première les activités de marchés et à la seconde celles de finance d'entreprise. Toutefois le terme de banque de financement et d'investissement (BFI) qui inclut les deux activités, tend à se généraliser. 

\subsubsection{Courtier}
Le courtier est le professionnel effectuant l'activité de courtage (brokerage pour les anglophones). Par son action, il sert d'intermédiaire pour une opération, le plus souvent financière, entre deux parties. \\

Il es important de différencier les différents types de courtier présent sur le marché : les sociétés de bourse (négocie les titres financiers), courtier indépendant (en général sur internet) et les services des banques. Les courtiers peuvent être réglé de manière forfaitaire (défini à l'avance) ou au contraire à l'acte, c'est-à-dire en fonction du nombre d'opérations qu'ils effectuent. \\

Le courtage est réglementé dans de nombreux pays, afin de protéger les intervenants sur le marché.


\subsubsection{AMF}
L'Autorité des marchés financiers (AMF) est une autorité publique française indépendante créée en 2003, dont les missions principales sont de protéger l'épargne investie dans les instruments financer, veiller à que les investisseurs aient un accès à l'information et de veiller au bon fonctionnement des marchés d'instruments financiers. \\

Elle aide à la régulation du marché aussi bien européen qu'international. L'AMF a les pouvoirs suivants : éditer les règles, surveiller les acteurs et les produits soumis à son contrôle et dispose d'un pouvoir de sanction par exemple. 