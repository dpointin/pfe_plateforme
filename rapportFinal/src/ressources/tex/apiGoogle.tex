\section{Affichage et visualisation des cours}
%http://www.simpleentrepreneur.com/2007/06/07/16-librairies-et-scripts-pour-generer-des-graphiques-sur-internet/
%http://www.fusioncharts.com/goodies/fusioncharts-free/
%http://docs.fusioncharts.com/free/#_ga=1.124638285.613042136.1459322699
%https://fr.wikipedia.org/wiki/Adobe_Flash

L'un des objectifs de notre projet étant de fournir à l'utilisateur une IHM pour visualiser les cours ainsi que des indicateurs financiers via l'analyse technique que nous verrons dans le chapitre suivant, nous avons besoin d'afficher des graphes et autres diagrammes.
Par exemple, nous pourrions vouloir afficher l'historique d'un cours sur une période ou encore visualiser la répartition des actifs d'un portefeuille sous forme diagramme en camembert. Le graphique doit donc être afficher sur une page web sachant que nous utilisons une plateforme Java EE.\\

Pour cela, plusieurs possibilités s'offraient à nous :
\begin{itemize}
 \item Publier les graphiques en 'Flash',
 \item Générer des graphes sous forme d'image puis les afficher,
 \item Utiliser une générateur Javascript permettant l'affichage de graphiques.
\end{itemize}

Nous allons présenter un exemple de chacune de ces possibilités que nous aurions pu utiliser puis nous expliquerons notre choix.

\subsection{Graphiques en Flash : FusionCharts Free}


\subsection{Images de graphes : JFreeChart}


\subsection{Javascript : API Google Chart}


\subsection{Notre choix}