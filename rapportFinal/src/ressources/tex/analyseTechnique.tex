\section{Analyse Technique}

\subsection{Présentation}

\subsubsection{Histoire de l'analyse technique}
Les sources de l'analyse technique remonte au XVIIIème siècle quand les Japonais essayaient d'anticiper l'évolution des cours du riz. Ensuite, on retrouve trace de l'analyse technique au début du XXème siècle grâce aux recherches de Richard Dow. Au début, l'analyse technique était purement graphique mais désormais on retrouve une grande part d'outil mathématiques. \\

A partir des années 30, Ralph Nelson Elliot a mis en évidence les fameuses vagues d'Elliot. Nous retrouvons également d'autres grands noms associés à l'analyse technique tels que Steve Nison, réputé pour la méthode des chandeliers, Stan Weinstein, pour les moyennes mobiles et John Bollinger. \\


\subsubsection{Définition}
Tentant de définir l’analyse technique, John Murphy disait : « L’analyse technique est l’étude de l’évolution d’un marché, principalement sur la base de graphiques, dans le but de prévoir les futures tendances ». L'analyse technique correspond à l'étude des graphiques des cours de la bourse ainsi que de divers indicateurs déduits de ces cours. Grâce à cette étude, le but est d'essayer d'anticiper l'évolution future des cours. \\

L'analyse technique peut s'appliquer à tout types de marchés : indices, actions, taux, matières premières. Les mêmes méthodes pouvant être appliqués dans tous les cas. 

L'analyse technique repose sur trois hypothèses fondamentales :
\begin{itemize}
\item Le prix intègre toute l'information disponible
\item Les prix évoluent en tendance
\item L'histoire se répète
\end{itemize}


\subsubsection{Différents courants}

Au fur et à mesure de l'histoire, l'analyse technique a évolué et s'est perfectionné. C'est ainsi que l'on peut distinguer quatre principaux courants dans l'analyse technique moderne :
\begin{itemize}
\item \textbf{Analyse technique chartiste} repose sur l'étude des cours et historique avec la recherche de motifs se répétant.  
\item \textbf{Analyse technique statistique} repose essentiellement sur l'étude de la modélisation de l'évolution des cours.
\item \textbf{Les vagues d'Elliot} cherchent à décomposer le cours comme étant une fractale.
\item \textbf{Le market profile} consiste en une étude statistique des cours, repose sur l'hypothèse d'une loi normale pour les cours. 
\end{itemize}

\subsection{Différentes outils}
Dans la partie précédente nous avons vu qu'au cours de l'histoire l'analyse technique n'avait cessé d'évoluer au fur et à mesure des découvertes. Nous allons présenté dans cette partie quatre outils d'analyse technique que nous allons mettre à disposition de l'utilisateur dans notre jeu. \\

\subsubsection{Les chandeliers}

\subsubsection{Les moyennes mobiles}

\subsubsection{Les Bollinger}

\subsubsection{Les volumes}


